\documentclass{article}
\usepackage[utf8]{inputenc}
\usepackage[margin=1.2in]{geometry}
\usepackage{hyperref}

\usepackage{listings}
\usepackage{xcolor}

\definecolor{codegreen}{rgb}{0,0.6,0}
\definecolor{codegray}{rgb}{0.5,0.5,0.5}
\definecolor{codepurple}{rgb}{0.58,0,0.82}
\definecolor{backcolour}{rgb}{.95,.95,.95}

\lstdefinestyle{mystyle}{
    backgroundcolor=\color{backcolour},   
    commentstyle=\color{codegreen},
    %keywordstyle=\color{magenta},
    numberstyle=\tiny\color{codegray},
    %stringstyle=\color{codepurple},
    %basicstyle=\ttfamily\footnotesize,
    breakatwhitespace=false,         
    breaklines=true,                 
    captionpos=b,                    
    keepspaces=true,                 
    numbers=left,                    
    numbersep=5pt,                  
    showspaces=false,                
    showstringspaces=false,
    showtabs=false,                  
    tabsize=2
}

\lstset{style=mystyle}


\usepackage{tikz}
\usetikzlibrary{positioning}

\usepackage{natbib}
\usepackage{graphicx}
\usepackage{amsmath}

\title{\vspace{-2 cm}Universidade Federal de Ouro Preto \\ PCC104 - Projeto e Análise de Algoritmos \\ Prova 2}
\author{Prof. Rodrigo Silva}
%\date{}


\begin{document}

\maketitle

\section*{Orientações}

\begin{itemize}
    \item É obrigatória a entrega do código fonte dos algoritmos implementados. Provas sem os códigos fonte não serão corrigidas e terão nota 0.
    \item O código não deve conter nenhuma informação/comentário que auxilie na análise de complexidade do mesmo. 
    \item A avaliação do código apresentado entra na avaliação das questões relacionadas. 
    \item Simplificações feitas na análise de custo dos algoritmos devem ser indicadas e justificadas.
\end{itemize}

\section*{Questões}

\begin{enumerate}
    
    \item Considere a implementação de árvore de busca binária abaixo:
        \lstinputlisting[language=Python]{BST.py}

        \begin{enumerate}
            \item (1 pt) Apresente a árvore gerada quando adicionamos os elementos $[6,4,7,10,9,8,2,5]$ nesta ordem.
            \item (1 pt) Apresente a lista de elementos retornados pela sua implementação do caminhamento posorder.
            \item (1 pt) Apresente a lista de elementos retornados pela sua implementação do caminhamento inorder.
        \end{enumerate}

    \item Considere a sua implementação do algoritmo de divisão e conquista que encontra a posição do maior elemento de um array.
    \begin{enumerate}
        \item (1 pt) Apresente a expressão matemática que define o custo em termos do número de comparações. Explique o que representa e de onde saiu cada um dos termos da sua expressão. 
        \item (1 pt) Resolva a expressão para encontrar o custo do algoritmo.
        \item (1 pt) Determine a classe de complexidade em notação $O$ ou $\Theta$ utilizando algum método formal. 
    \end{enumerate}

    \item Considere a sua implementação do quicksort.
    \begin{enumerate}
        \item (1 pt) Apresente a expressão matemática que define o custo em termos do número de comparações para o melhor caso. Explique o que representa e de onde saiu cada um dos termos da sua expressão. 
        \item (1 pt) Resolva a expressão para encontrar o custo do algoritmo.
        \item (1 pt) Determine a classe de complexidade em notação $O$ ou $\Theta$ utilizando algum método formal.
        \item (1 pt) Mostre a execução do seu algoritmo no array $[3,5,7,8,1,2,4]$.
    \end{enumerate}
   
\end{enumerate}



%\bibliographystyle{plain}
%\bibliography{references}
\end{document}
