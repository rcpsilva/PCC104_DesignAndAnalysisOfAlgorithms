\documentclass{article}
\usepackage[utf8]{inputenc}
\usepackage[margin=1.2in]{geometry}
\usepackage{hyperref}

\usepackage{listings}
\usepackage{xcolor}

\definecolor{codegreen}{rgb}{0,0.6,0}
\definecolor{codegray}{rgb}{0.5,0.5,0.5}
\definecolor{codepurple}{rgb}{0.58,0,0.82}
\definecolor{backcolour}{rgb}{0.95,0.95,0.92}

\lstdefinestyle{mystyle}{
    backgroundcolor=\color{backcolour},   
    commentstyle=\color{codegreen},
    keywordstyle=\color{magenta},
    numberstyle=\tiny\color{codegray},
    stringstyle=\color{codepurple},
    basicstyle=\ttfamily\footnotesize,
    breakatwhitespace=false,         
    breaklines=true,                 
    captionpos=b,                    
    keepspaces=true,                 
    numbers=left,                    
    numbersep=5pt,                  
    showspaces=false,                
    showstringspaces=false,
    showtabs=false,                  
    tabsize=2
}

\lstset{style=mystyle}


\usepackage{tikz}
\usetikzlibrary{positioning}

\usepackage{natbib}
\usepackage{graphicx}
\usepackage{amsmath}

\title{\vspace{-2 cm}Universidade Federal de Ouro Preto \\ PCC104 - Projeto e Análise de Algoritmos \\ Branch and Bound}
\author{Prof. Rodrigo Silva}
%\date{}


\begin{document}

\maketitle

\section*{Leitura Recomendada}

\begin{itemize}
    \item Capítulo 12 (12.4) - \textit{Introduction to the Design and Analysis of Algorithms (3rd Edition)} - Anany Levitin 
\end{itemize}


\section{Atividades}

\begin{enumerate}
    \item Implemente um algoritmo para a reolução do problema do caixeiro viajante baseado em força bruta.
    \item Implemente um algoritmo para a reolução do problema do caixeiro viajante baseado em branch and bound. 
    \item Compare as duas abordagens em termos do número operações para instâncias aleatórias de 3 tamanhos diferentes. Para cada tamanho , gere 10 instâncias. Ao final, gere um boxplot com os resultados.
\end{enumerate}

Para cada implementação, apresentar a análise de complexidade de tempo do algoritmo. Esta análise deverá conter:

\begin{itemize}
    \item Definição clara e explícita da operação básica.
    \item Expressão matemática que define o custo do algoritmo (relação de recorrência para recursivos ou somatórios para iterativos) 
    \item Uma reflexão sobre melhor caso, pior caso e caso médio.
    \item Cálculo da função de custo (quando possível, utilizar o teorema mestre para verificar o cálculo).
    \item Indicação da classe de eficiência ($O$ ou $\Theta$). A indicação da classe, deve ser justificada. Você pode provar pela definição, pelo limite, teorema mestre ou utilizar os resultados demonstrados em aula.
\end{itemize}

%\bibliographystyle{plain}
%\bibliography{references}
\end{document}
