\documentclass{article}
\usepackage[utf8]{inputenc}
\usepackage[margin=1.2in]{geometry}
\usepackage{hyperref}

\usepackage{listings}
\usepackage{xcolor}

\definecolor{codegreen}{rgb}{0,0.6,0}
\definecolor{codegray}{rgb}{0.5,0.5,0.5}
\definecolor{codepurple}{rgb}{0.58,0,0.82}
\definecolor{backcolour}{rgb}{0.95,0.95,0.92}

\lstdefinestyle{mystyle}{
    backgroundcolor=\color{white},   
    commentstyle=\itshape\color{green!40!black},
    keywordstyle=\color{blue},
    numberstyle=\tiny\color{codegray},
    stringstyle=\color{codepurple},
    basicstyle=\ttfamily\footnotesize,
    breakatwhitespace=false,         
    breaklines=true,                 
    captionpos=b,                    
    keepspaces=true,                 
    numbers=left,                    
    numbersep=5pt,                  
    showspaces=false,                
    showstringspaces=false,
    showtabs=false,                  
    tabsize=2
}

\lstset{style=mystyle}


\usepackage{tikz}
\usetikzlibrary{positioning}

\usepackage{natbib}
\usepackage{graphicx}
\usepackage{amsmath}

\title{\vspace{-2 cm}Universidade Federal de Ouro Preto \\ Lecture Notes \\ Graph Representation}
\author{Prof. Rodrigo Silva}
%\date{}


\begin{document}

\maketitle

\section{Graphs in C++}

\subsection{Adjacency list}

\lstinputlisting[language=C++,caption=Example 1 - Graph as Adjacency List C++ (Iteractive)]{graph_adj_list.cpp}

\subsection{Adjacency matrix}

\lstinputlisting[language=C++,caption=Example 1 - Graph as Adjacency List C++]{graph_matrix.cpp}

\section{Graphs in Python}

\subsection{Adjacency list}

\lstinputlisting[language=Python,caption=Example 1 - Graph as Adjacency List Python]{graph_adj_list.py}

\subsection{Adjacency matrix}

\lstinputlisting[language=Python,caption=Example 1 - Graph as Matrix Python]{graph_matrix.py}


\end{document}

