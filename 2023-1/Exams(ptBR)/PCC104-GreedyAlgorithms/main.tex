\documentclass{article}
\usepackage[utf8]{inputenc}
\usepackage[margin=1.2in]{geometry}
\usepackage{hyperref}

\usepackage{listings}
\usepackage{xcolor}

\definecolor{codegreen}{rgb}{0,0.6,0}
\definecolor{codegray}{rgb}{0.5,0.5,0.5}
\definecolor{codepurple}{rgb}{0.58,0,0.82}
\definecolor{backcolour}{rgb}{0.95,0.95,0.92}

\lstdefinestyle{mystyle}{
    backgroundcolor=\color{backcolour},   
    commentstyle=\color{codegreen},
    keywordstyle=\color{magenta},
    numberstyle=\tiny\color{codegray},
    stringstyle=\color{codepurple},
    basicstyle=\ttfamily\footnotesize,
    breakatwhitespace=false,         
    breaklines=true,                 
    captionpos=b,                    
    keepspaces=true,                 
    numbers=left,                    
    numbersep=5pt,                  
    showspaces=false,                
    showstringspaces=false,
    showtabs=false,                  
    tabsize=2
}

\lstset{style=mystyle}


\usepackage{tikz}
\usetikzlibrary{positioning}

\usepackage{natbib}
\usepackage{graphicx}
\usepackage{amsmath}

\title{\vspace{-2 cm}Universidade Federal de Ouro Preto \\ PCC104 - Projeto e Análise de Algoritmos \\ Prova - Algoritmos Gulosos}
\author{Prof. Rodrigo Silva}
%\date{}


\begin{document}

\maketitle

\section*{Orientações}

\begin{itemize}
    \item Esta prova tem 4 questões.
    \item É obrigatória a entrega do código fonte dos algoritmos gulosos. Provas sem os códigos fonte não serão corrigidas e terão nota 0.
    \item A avaliação do código apresentado entre na avaliação das questões relacionadas. 
\end{itemize}

\section*{Questões}

\begin{enumerate}
    \item Quando devemos utilizar algoritmos gulosos?

    \item Considere o pseudo-código do algoritmo de Prim abaixo, a sua própria implementação e responda:
    
    \begin{figure}[!ht]
        \centering
        \includegraphics*[width=0.8\textwidth]{prim.png}
        \caption{Algoritmo de Prim}
    \end{figure}

    \begin{enumerate}
        \item Como você implementou a operação que encontra a aresta de menor peso? Indique no seu código e apresente a análise de complexidade completa desta operação.
        \item Apresente a análise de complexidade da sua implementação do algoritmo de Prim.
        \item Discuta a possibilidade de melhorar o seu código. Caso não seja possível melhora, explique o por quê. 
    \end{enumerate}

    \pagebreak

    \item Considere o pseudo-código do algoritmo de Kruskall abaixo, a sua própria implementação e responda:
    
    \begin{figure}[!ht]
        \centering
        \includegraphics*[width=0.8\textwidth]{kruskall.png}
        \caption{Algoritmo de Kruskall}
    \end{figure}

    \begin{enumerate}
        \item Como você implementou a operação que verifica se uma aresta faz um ciclo? Indique no seu código e apresente a análise de complexidade completa desta operação.
        \item Apresente a análise de complexidade da sua implementação do algoritmo de Kruskall.
        \item Discuta a possibilidade de melhorar o seu código. Caso não seja possível melhora, explique o por quê. 
    \end{enumerate}

    \item Considere o problema da árvore geradora mínima definido como um problema de decisão. Ou seja, ´´Existe uma árvore geradora com custo menor ou igual a X?''. Responda:
    
    \begin{enumerate}
        \item A qual classe de problemas este problema pertence, P, NP ou NP-completo? Justifique.
        \item O que precisaríamos fazer para provar que este problema pertence à classe NP?
        \item O que precisaríamos fazer para provar que este problema pertence à classe NP-completo?
    \end{enumerate}

\end{enumerate}


%\bibliographystyle{plain}
%\bibliography{references}
\end{document}
