\documentclass{article}
\usepackage[utf8]{inputenc}
\usepackage[margin=1.2in]{geometry}
\usepackage{hyperref}

\usepackage{listings}
\usepackage{xcolor}

\definecolor{codegreen}{rgb}{0,0.6,0}
\definecolor{codegray}{rgb}{0.5,0.5,0.5}
\definecolor{codepurple}{rgb}{0.58,0,0.82}
\definecolor{backcolour}{rgb}{1,1,1}

\lstdefinestyle{mystyle}{
    backgroundcolor=\color{backcolour},   
    commentstyle=\color{codegreen},
    keywordstyle=\color{magenta},
    numberstyle=\tiny\color{codegray},
    stringstyle=\color{codepurple},
    basicstyle=\ttfamily\small,
    breakatwhitespace=false,         
    breaklines=true,                 
    captionpos=b,                    
    keepspaces=true,                 
    numbers=left,                    
    numbersep=5pt,                  
    showspaces=false,                
    showstringspaces=false,
    showtabs=false,                  
    tabsize=2
}

\lstset{style=mystyle}


\usepackage{tikz}
\usetikzlibrary{positioning}

\usepackage{natbib}
\usepackage{graphicx}
\usepackage{amsmath}
\usepackage[utf8]{inputenc}
\usepackage{listings}


\title{\vspace{-2 cm}Universidade Federal de Ouro Preto \\ PCC104 - Projeto e Análise de Algoritmos \\ Prova - Algoritmos Gulosos}
\author{Prof. Rodrigo Silva}
%\date{}


\begin{document}

\maketitle

\section*{Orientações}

\begin{itemize}
    \item É obrigatória a entrega do código fonte da prática de branch and bound. Provas sem os códigos fonte não serão corrigidas e terão nota 0.
    \item A avaliação do código apresentado entra na avaliação das questões relacionadas. 
\end{itemize}


\section*{Questões}

\begin{enumerate}
    

\item \textit{Análise de Algoritmo Iterativo Simples}

Considerando o algoritmo de ordenação apresentado na Figura \ref{fig:alg1_pyhton} (para a versão C++, veja \ref{fig:alg1_cpp}), que implementa o método de "Bubble Sort". Apresente a análise completa e detalhada deste algoritmo? 

\item \textit{Análise de Algoritmo Recursivo Simples}

Veja algoritmo de busca binária apresentado na figura \ref{fig:alg2_python}  (aara a versão em C++ veja \ref{fig:alg2_cpp}) e responda:

\begin{enumerate}
\item Apresente a análise completa e detalhada da complexidade de tempo do pior caso deste algoritmo. 
\item Apresente a análise completa e detalhada da complexidade de tempo do melhor caso deste algoritmo. 
\end{enumerate}

\item \textit{Análise e Perguntas Teóricas Sobre o Branch and Bound}

\begin{enumerate}
\item  Explique o conceito de branch and bound e sua aplicação na resolução de problemas de otimização. 
\item  Compare o branch and bound com o método de força bruta. Em quais cenários cada um seria preferível e por quê? 
\item  Como a estratégia de branch and bound pode impactar o custo computacional da resolução de um problema? 
\end{enumerate}

\item \textit{Perguntas Teóricas Sobre Classes de Problemas (P, NP, NP-completo)}

É possível que P = NP? Explique sua resposta, discutindo as implicações se P fosse de fato igual a NP. 

\end{enumerate}


\begin{figure} [!ht]
    \begin{lstlisting}[language=Python]
        def bubble_sort(lista):
            for i in range(len(lista)):
                for j in range(0, len(lista) - i - 1):
                    if lista[j] > lista[j + 1]:
                        lista[j], lista[j + 1] = lista[j + 1], lista[j]
        \end{lstlisting}
        \label{fig:alg1_pyhton}
        \caption{Algoritmo 1 - Versão Python}
\end{figure}

\begin{figure} [!ht]
    \begin{lstlisting}[language=c++]
    void bubble_sort(vector<int>& lista) {
        int i, j;
        bool swapped;
        int n = lista.size();

        for (i = 0; i < n-1; i++) {
            swapped = false;
            for (j = 0; j < n-i-1; j++) {
                if (lista[j] > lista[j+1]) {
                    int temp = lista[j];
                    lista[j] = lista[j+1];
                    lista[j+1] = temp;
                    swapped = true;
                }
            }
            if (swapped == false)
                break;
        }
}
    \end{lstlisting}
        \label{fig:alg1_cpp}
        \caption{Algoritmo 1 - Versão C++}
\end{figure}

\begin{figure}

\begin{lstlisting}[language=Python]
    def binary_search(array, low, high, target):
        if high >= low:
            mid = (high + low) // 2
            if array[mid] == target:
                return mid
            elif array[mid] > target:
                return binary_search(array, low, mid - 1, target)
            else:
                return binary_search(array, mid + 1, high, target)
        else:
            return -1
    \end{lstlisting}
    \label{fig:alg2_python}
    \caption{Algortimo 2 - Versão Python}
\end{figure}

\begin{figure}
\begin{lstlisting}[language=Python]
    int binary_search(vector<int>& array, int low, int high, int target) {
        if (high >= low) {
            int mid = low + (high - low) / 2;  
            if (array[mid] == target)
                return mid;
            else if (array[mid] > target)
                return binary_search(array, low, mid - 1, target);
            else
                return binary_search(array, mid + 1, high, target);
        } else {
            return -1;
        }
    }
    \end{lstlisting}
    \label{fig:alg2_cpp}
    \caption{Algortimo 2 - Versão Python}
\end{figure}

%\bibliographystyle{plain}
%\bibliography{references}
\end{document}
