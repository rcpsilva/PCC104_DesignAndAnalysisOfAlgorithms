\documentclass{article}
\usepackage{listings}
\usepackage{amsmath}
\usepackage{geometry}
\usepackage{url}
\geometry{a4paper, margin=1in}

\title{Recursion, Dynamic Programming, and Binary Search}
\author{}
\date{}

\begin{document}

\maketitle

\section*{Instructions}

\begin{itemize}
    \item The references fir this problem set are Chapters 9, 10, and 11 of the textbook and Levitin's book.
    \item The textbook is available at \url{https://github.com/rcpsilva/PCC104_DesignAndAnalysisOfAlgorithms/blob/main/2025-1/Course%20Material/A%20First%20Course%20on%20Data%20Structures%20in%20Python.pdf}
    \item Presente the complexity analysis of your solutions.
\end{itemize}

\section*{Recursion (Chapter 9)}

\begin{enumerate}
    \item \textbf{Recursive Maximum Finder}

    Write a recursive function \texttt{find\_max(L)} that returns the maximum element in a list \texttt{L}.

    \textbf{Example:}
    \begin{lstlisting}[language=Python]
    find_max([1, 5, 3, 9, 2])  # Output: 9
    \end{lstlisting}

    \item \textbf{Recursive String Reversal}

    Implement a recursive function \texttt{reverse(s)} that returns the reversed version of the string \texttt{s}.

    \textbf{Example:}
    \begin{lstlisting}[language=Python]
    reverse("python")  # Output: "nohtyp"
    \end{lstlisting}

    \item \textbf{Sum of Digits}

    Write a recursive function \texttt{sum\_digits(n)} that returns the sum of the digits of an integer \texttt{n}.

    \textbf{Example:}
    \begin{lstlisting}[language=Python]
    sum_digits(1234)  # Output: 10
    \end{lstlisting}

    \item \textbf{Palindrome Check}

    Write a recursive function \texttt{is\_palindrome(s)} that returns \texttt{True} if the string \texttt{s} is a palindrome and \texttt{False} otherwise.

    \textbf{Example:}
    \begin{lstlisting}[language=Python]
    is_palindrome("racecar")  # Output: True
    is_palindrome("hello")    # Output: False
    \end{lstlisting}
\end{enumerate}

\section*{Dynamic Programming (Chapter 10)}

\begin{enumerate}
    \setcounter{enumi}{4}
    \item \textbf{Fibonacci with Memoization}

    Implement a memoized version of the Fibonacci sequence. The function \texttt{fib(n)} should return the $n$th Fibonacci number.

    \textbf{Example:}
    \begin{lstlisting}[language=Python]
    fib(10)  # Output: 55
    \end{lstlisting}

    \item \textbf{Minimum Coin Change}

    Given coins of certain denominations and a total amount, write a function \texttt{min\_coins(coins, amount)} that computes the minimum number of coins needed to make the amount.

    \textbf{Example:}
    \begin{lstlisting}[language=Python]
    min_coins([1, 3, 4], 6)  # Output: 2 (3 + 3)
    \end{lstlisting}

    \item \textbf{Longest Common Subsequence (LCS)}

    Write a function \texttt{lcs(X, Y)} that returns the length of the longest common subsequence of two strings \texttt{X} and \texttt{Y}.

    \textbf{Example:}
    \begin{lstlisting}[language=Python]
    lcs("abcde", "ace")  # Output: 3
    \end{lstlisting}

    \item \textbf{0/1 Knapsack Problem}

    Given weights and values of $n$ items, write a function \texttt{knapsack(W, weights, values)} to determine the maximum value that can be put in a knapsack of capacity $W$.

    \textbf{Example:}
    \begin{lstlisting}[language=Python]
    knapsack(50, [10, 20, 30], [60, 100, 120])  # Output: 220
    \end{lstlisting}
\end{enumerate}

\section*{Binary Search (Chapter 11)}

\begin{enumerate}
    \setcounter{enumi}{8}
    \item \textbf{Binary Search Implementation}

    Implement a recursive function \texttt{binary\_search(L, x)} that returns \texttt{True} if \texttt{x} is in the sorted list \texttt{L}, and \texttt{False} otherwise.

    \textbf{Example:}
    \begin{lstlisting}[language=Python]
    binary_search([1, 3, 5, 7, 9], 3)  # Output: True
    binary_search([1, 3, 5, 7, 9], 4)  # Output: False
    \end{lstlisting}

    \item \textbf{First Occurrence in Sorted Array}

    Modify your binary search to find the index of the first occurrence of a number in a sorted list with duplicates.

    \textbf{Example:}
    \begin{lstlisting}[language=Python]
    first_occurrence([1, 2, 2, 2, 3, 4], 2)  # Output: 1
    \end{lstlisting}

    \item \textbf{Square Root Using Binary Search}

    Implement a function \texttt{sqrt\_binary(n)} that returns the integer square root of a non-negative integer $n$ using binary search.

    \textbf{Example:}
    \begin{lstlisting}[language=Python]
    sqrt_binary(10)  # Output: 3
    sqrt_binary(16)  # Output: 4
    \end{lstlisting}

    \item \textbf{Peak Element Finder}

    Write a function \texttt{find\_peak(L)} that returns an index of a peak element using a binary search-like approach. An element is a peak if it is greater than or equal to its neighbors.

    \textbf{Example:}
    \begin{lstlisting}[language=Python]
    find_peak([1, 3, 20, 4, 1, 0])  # Output: 2 (index of 20)
    \end{lstlisting}
\end{enumerate}

\end{document}
