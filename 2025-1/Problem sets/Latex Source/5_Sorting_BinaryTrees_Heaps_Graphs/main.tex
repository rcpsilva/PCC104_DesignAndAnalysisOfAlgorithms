\documentclass{article}
\usepackage{listings}
\usepackage{amsmath}
\usepackage{geometry}
\usepackage{url}
\usepackage{hyperref} 
\geometry{a4paper, margin=1in}

\title{Sorting, Binary Trees, Priority Queues, and Graphs}
\author{}
\date{}

\begin{document}

\maketitle

\section*{Sorting}

\begin{enumerate}
    \item Implement the \textbf{Selection Sort} algorithm and analyze its best-case and worst-case time complexity.
    \item Implement the \textbf{Insertion Sort} algorithm and analyze its best-case and worst-case time complexity.
    \item Implement the \textbf{Quick Sort} algorithm and analyze its best-case and worst-case time complexity.
\end{enumerate}

\section*{Binary Trees}

\begin{enumerate}
    \setcounter{enumi}{3}
    \item Implement the missing methods in the code available at:\\
    \url{https://github.com/rcpsilva/PCC104_DesignAndAnalysisOfAlgorithms/blob/main/2025-1/Problem%20sets/Latex%20Source/5_Sorting_BinaryTrees_Heaps_Graphs/binary_search_tree.py}\\
    For each implemented method, present its time complexity analysis. You may define auxiliary methods if needed.
\end{enumerate}

\section*{Heaps}

\begin{enumerate}
    \setcounter{enumi}{4}
    \item Using the base code available at: \\
    \url{https://github.com/rcpsilva/PCC104_DesignAndAnalysisOfAlgorithms/blob/main/2025-1/Problem%20sets/Latex%20Source/5_Sorting_BinaryTrees_Heaps_Graphs/heap.py}\\
    Implement a \textbf{min-heap} using a Python list as the underlying data structure. Ensure that the heap supports insertion and removal of the minimum element.
\end{enumerate}

\section*{Graphs}

\begin{enumerate}
    \setcounter{enumi}{5}
    \item Implement the \textbf{Depth-First Search (DFS)} and \textbf{Breadth-First Search (BFS)} algorithms for a graph represented using an adjacency list. Analyze the time and space complexity of both algorithms in terms of the graph's depth $h$ and average branching factor $b$.
    
    \item Implement the \textbf{DFS}, \textbf{BFS}, and \textbf{A$^*$} algorithms to find a path in a maze. The maze is represented as an $m \times n$ matrix, where each cell may contain:
    
    \begin{itemize}
        \item \texttt{0}: free space;
        \item \texttt{1}: wall;
        \item \texttt{s}: starting position;
        \item \texttt{g}: goal (maze exit).
    \end{itemize}
    
    Your implementation should return a valid path (if one exists) from the start to the goal position.
\end{enumerate}

\end{document}
