\documentclass{article}
\usepackage{listings}
\usepackage{amsmath}
\usepackage{geometry}
\geometry{a4paper, margin=1in}

\title{Exercises on Basic Python, Object-Oriented Programming, and Testing}
\author{}
\date{}

\begin{document}

\maketitle

\section*{Basic Python (Chapter 2)}

\begin{enumerate}
    \item Compute the result and explain how Python evaluates the expressions below:
    \begin{itemize}
        \item \texttt{5 + 3 * 2}
        \item \texttt{10 / 3}
        \item \texttt{10 // 3}
        \item \texttt{10 \% 3}
        \item \texttt{5 ** 2}
    \end{itemize}
    
    \item What will be the output of the following Python program? Explain why.
    
    \begin{lstlisting}[language=Python]
    x = 3.5
    y = int(x)
    print(y, type(y))
    \end{lstlisting}
    
    \item Lists:

    \begin{enumerate}
        \item Create a list with the first five prime numbers.\\
        \textbf{Example:} Output: \texttt{[2, 3, 5, 7, 11]}
        
        \item Write a function to remove all even numbers from a given list.\\
        \textbf{Example:} \\
        Input: \texttt{[1,2,3,4,5,6]} \\
        Output: \texttt{[1,3,5]}
        
        \item Implement a function that finds the second largest number in a list.\\
        \textbf{Example:} \\
        Input: \texttt{[10,20,4,45,99]} \\
        Output: \texttt{45}
        
        \item Write a function that flattens a nested list.\\
        \textbf{Example:} \\
        Input: \texttt{[[1,2],[3,4],[5]]} \\
        Output: \texttt{[1,2,3,4,5]}
        
        \item Write a function that rotates a list to the right by a given number of positions.\\
        \textbf{Example:} \\
        Input: \texttt{[1,2,3,4,5]}, Rotate by 2 \\
        Output: \texttt{[4,5,1,2,3]}    
        
        \item Write a function to count the occurrences of each element in a list.\\
        \textbf{Example:} \\
        Input: \texttt{[1,1,2,3,3,3,4]} \\
        Output: \texttt{\{1:2, 2:1, 3:3, 4:1\}}
        
    \end{enumerate}

    \item Tuples:

    \begin{enumerate}
        \item Convert a list of tuples into a dictionary.\\
        \textbf{Example:} \\
        Input: \texttt{[(1,'a'), (2,'b'), (3,'c')]} \\
        Output: \texttt{\{1:'a', 2:'b', 3:'c'\}}
        
        \item Write a function that swaps the first and last elements of a tuple.\\
        \textbf{Example:} \\
        Input: \texttt{(1,2,3,4,5)} \\
        Output: \texttt{(5,2,3,4,1)}
        
        \item Write a function that finds the maximum and minimum values in a tuple.\\
        \textbf{Example:} \\
        Input: \texttt{(4, 7, 1, 9)} \\
        Output: \texttt{(9,1)}
        
        \item Write a function that converts a tuple of numbers into a single concatenated string.\\
        \textbf{Example:} \\
        Input: \texttt{(1,2,3,4)} \\
        Output: \texttt{"1234"}
        
    \end{enumerate}
        
    \item Dictionaries

        \begin{enumerate}
            
            \item Write a function that merges two dictionaries, summing values of common keys.\\
            \textbf{Example:} Input: \texttt{\{'a':1, 'b':2\}}, \texttt{\{'b':3, 'c':4\}}\\
            Output: \texttt{\{'a':1, 'b':5, 'c':4\}}
            
            \item Write a function that inverts a dictionary (keys become values and vice versa).\\
            \textbf{Example:} Input: \texttt{\{'a': 1, 'b': 2\}}\\
            Output: \texttt{\{1: 'a', 2: 'b'\}}
            
            \item Write a function to find the most frequently occurring value in a dictionary.\\
            \textbf{Example:} Input: \texttt{\{'a': 3, 'b': 2, 'c': 3\}}\\
            Output: \texttt{3}
            
            \item Write a function that groups words by their first letter from a given list. The function should return a dictionary where the keys are the first letters and the values are lists of words.\\
            \textbf{Example:} Input: \texttt{["apple", "banana", "apricot", "blueberry", "cherry"]}\\
            Output: \texttt{\{"a": ["apple", "apricot"], "b": ["banana", "blueberry"], "c": ["cherry"]\}}
            
            \item Write a function that finds the key associated with the highest value in a dictionary.\\
            \textbf{Example:} Input: \texttt{\{'a': 10, 'b': 25, 'c': 17\}}\\
            Output: \texttt{'b'}
        \end{enumerate}
    
        \item Sets:

        \begin{enumerate}
            \item Write a function that returns the union, intersection, and difference of two sets.\\
            \textbf{Example:} \\
            Input: \texttt{\{1,2,3\}}, \texttt{\{3,4,5\}} \\
            Output: \\
            Union: \texttt{\{1,2,3,4,5\}} \\
            Intersection: \texttt{\{3\}} \\
            Difference (Set 1 - Set 2): \texttt{\{1,2\}} \\
            Difference (Set 2 - Set 1): \texttt{\{4,5\}}
            
            \item Write a function that finds the symmetric difference between two sets.\\
            \textbf{Example:} \\
            Input: \texttt{\{1,2,3\}}, \texttt{\{2,3,4\}} \\
            Output: \texttt{\{1,4\}}
            
            \item Write a function to check if two sets are disjoint.\\
            \textbf{Example:} \\
            Input: \texttt{\{1,2,3\}}, \texttt{\{4,5,6\}} \\
            Output: \texttt{True} \\
            \\
            Input: \texttt{\{1,2,3\}}, \texttt{\{3,4,5\}} \\
            Output: \texttt{False}
            
        \end{enumerate}
        
\end{enumerate}

\section*{Object-Oriented Programming (Chapter 3)}

\begin{enumerate}
    \item Define a class \texttt{Rectangle} in Python with:
    \begin{itemize}
        \item Attributes: \texttt{length} and \texttt{width}.
        \item A method \texttt{area()} that returns the area of the rectangle.
        \item A method \texttt{perimeter()} that returns the perimeter of the rectangle.
    \end{itemize}

    \item Define a class \texttt{Circle} with an attribute \texttt{radius}. Include methods:
    \begin{itemize}
        \item \texttt{area()} that calculates the area.
        \item \texttt{circumference()} that calculates the circumference.
    \end{itemize}
    \textbf{Example:} \texttt{Circle(5).area()} returns approximately 78.54.

    \item Define a class \texttt{Student} with attributes \texttt{name}, \texttt{age}, and \texttt{grades}. Include methods:
    \begin{itemize}
        \item \texttt{average()} returning the average of grades.
        \item \texttt{is\_passing()} returning \texttt{True} if the average grade is above a threshold (e.g., 60).
    \end{itemize}
    \textbf{Example:} \texttt{Student("Alice",20,[80,90]).average()} returns 85.

    \item Define a class \texttt{BankAccount} with attributes \texttt{balance} and methods \texttt{deposit()} and \texttt{withdraw()}.\\
    \textbf{Example:} After depositing 50 into an account initialized with 100, the balance is 150.

    \item Implement inheritance by defining a superclass \texttt{Vehicle} with attributes \texttt{make} and \texttt{model}. Create subclasses \texttt{Car} and \texttt{Bike} with additional attributes \texttt{doors} for \texttt{Car} and \texttt{type} for \texttt{Bike}.\\
    \textbf{Example:} \texttt{Car("Ford","Mustang",4)} creates a car object.
\end{enumerate}

\section*{Testing (Chapter 4)}

\begin{enumerate}
    \item Write a Python function \texttt{is\_positive(n)} that returns \texttt{True} if \texttt{n} is positive and \texttt{False} otherwise. Use an \texttt{assert} statement to test your function.\\
    \textbf{Example:} Input: \texttt{5}, Output: \texttt{True}; Input: \texttt{-3}, Output: \texttt{False}

    \item Write unit tests for the \texttt{Rectangle} class using Python's \texttt{unittest} framework.

    \item Explain the concept of Test-Driven Development (TDD) and illustrate it by writing tests first for a simple function that calculates the factorial of a number.

    \item Write tests that specifically check edge cases, incorrect usage, and error handling for a function that divides two numbers.

    \item Explain why tests should be maintained even after they pass successfully.
\end{enumerate}

\section*{Running Time Analysis (Chapter 5)}

\begin{enumerate}
    \item Implement a Python program that adds the first \( k \) natural numbers in two ways: using a loop and using the formula \( S = \frac{k(k+1)}{2} \). Measure and compare the running time of both implementations.\\
    \textbf{Example:} Input: \texttt{k=1000}, Output: Time for loop: \texttt{X ms}, Time for formula: \texttt{Y ms}
    
    \item Analyze the time complexity of common list operations such as appending, indexing, and slicing. Provide Python examples and justify the complexity of each operation.
    
    \item Consider a dictionary with \( n \) elements. Measure the execution time of inserting, deleting, and accessing elements. Compare your results to the expected theoretical complexities.
    
    \item Analyze the complexity of the exercises in the Basic Python section. Provide the worst-case time complexity for each function.
    
    \item Explain the Big-O notation concept and determine the worst-case time complexity of the following function:
    \begin{lstlisting}[language=Python]
    def mystery_function(n):
        total = 0
        for i in range(n):
            for j in range(i):
                total += j
        return total
    \end{lstlisting}
\end{enumerate}

\end{document}
