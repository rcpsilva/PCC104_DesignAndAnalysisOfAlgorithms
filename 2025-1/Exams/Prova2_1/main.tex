\documentclass[12pt]{article}
\usepackage[utf8]{inputenc}
\usepackage[brazil]{babel}
\usepackage{amsmath, amssymb}
\usepackage{geometry}
\geometry{a4paper, margin=2.5cm}
\title{PCC104 – Prova 2 (Versão B)}
\author{Universidade Federal de Ouro Preto}
\date{}

\begin{document}

\maketitle

\begin{enumerate}

\item (2pts) Resolva a seguinte recorrência assumindo que $n$ é potência de 2:

\[
T(n) = T\left(\frac{n}{2}\right) + n,\quad T(1) = 1
\]

\item (1pt) Apresente um algoritmo para o qual a recorrência da Questão 1 represente o número de operações realizadas no pior caso. Justifique sua escolha.

\item (1pt) Considere a função \texttt{sum\_digits(n)}, definida de forma recursiva para retornar a soma dos dígitos de um número inteiro positivo $n$. 

Mostre a árvore de chamadas da função ao executar \texttt{sum\_digits(742)}. 

\item (2pts) Considere sua implementação da função \texttt{is\_palindrome(s)}, que verifica se uma string é palíndroma de forma recursiva. Apresente a análise de complexidade no pior caso.

\item (4pts) *Considere um vetor $V$ de números inteiros distintos e com tamanho $n$, onde $n$ é potência de 2. Um número $x$ é chamado de \textit{dominante local} se $x > x_{\text{vizinho à esquerda}}$ e $x > x_{\text{vizinho à direita}}$.

Escreva um algoritmo \texttt{find\_local\_dominant(V)} baseado em divisão e conquista que retorna o índice de um elemento dominante local (caso exista).

\begin{itemize}
  \item Justifique a complexidade da sua solução.
  \item \textbf{Dica:} Avalie o elemento central de $V$. Caso ele não seja dominante local, em qual lado (esquerdo ou direito) você deveria continuar buscando? Use os vizinhos para decidir.
\end{itemize}


\item (4pts) *Uma string contém letras minúsculas e o caractere '\#', que representa um backspace (remove o último caractere inserido).

Exemplo: \texttt{"ab\#c"} $\rightarrow$ \texttt{"ac"}

Implemente um algoritmo \texttt{processa(s)} que retorna a string final após aplicar os backspaces. Use uma pilha.

Justifique a complexidade da sua solução.

\end{enumerate}

$*$ Você deve escolher ou a questão 5 ou a questão 6.

\end{document}
