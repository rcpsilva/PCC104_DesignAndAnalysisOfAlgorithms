\documentclass[12pt]{article}
\usepackage[utf8]{inputenc}
\usepackage[brazil]{babel}
\usepackage{amsmath, amssymb}
\usepackage{geometry}
\geometry{a4paper, margin=2.5cm}
\title{PCC104 – Prova 1}
\author{Universidade Federal de Ouro Preto}
\date{}

\begin{document}

\maketitle


\begin{enumerate}

\item  Explique, com suas próprias palavras, o que significam os seguintes termos:
\begin{itemize}
  \item Melhor caso
  \item Caso médio
  \item Pior caso
\end{itemize}

\item Considere o seguinte algoritmo recursivo:

\begin{verbatim}
def soma_n(n):
    if n == 0:
        return 0
    else:
        return n + soma_n(n-1)
\end{verbatim}

\begin{itemize}
  \item[a)] O que este algoritmo computa?
  \item[b)] Qual a relação de recorrência associada ao número de adições?
  \item[c)] Qual a complexidade de tempo no pior caso (use notação O)? Apresente os calculos necessários para chegar a sua resposta.
\end{itemize}

\item Analise a complexidade do algoritmo abaixo:

\begin{verbatim}
def imprime_pares(lista):
    for i in lista:
        for j in lista:
            if (i + j) % 2 == 0:
                print(i, j)
\end{verbatim}

\begin{itemize}
  \item[a)] Qual é a operação básica?
  \item[b)] Quantas vezes ela é executada (em termos de n = len(lista))?
  \item[c)] Qual a complexidade de tempo no pior caso?  Apresente os calculos necessários para chegar a sua resposta.
\end{itemize}

\item Resolva a seguinte relação de recorrência:

\[ x(n) = x(n-1) + n, \quad x(0) = 0 \]

Encontre a solução fechada e a complexidade de tempo do algoritmo correspondente.

\item Resolva a seguinte relação de recorrência:

\[ x(n) = 2x(n/2) + 1, \quad x(1) = 1 \]

Encontre a solução fechada e a complexidade de tempo do algoritmo correspondente.

\item Prove que todo polinômio de grau $k$ do tipo $p(n) = a_k n^k + \dots + a_0$, com $a_i > 0$, pertence a $\Theta(n^k)$.

\item Implemente uma função que receba uma lista de inteiros e retorne um dicionário com o número de ocorrências de cada valor.

Exemplo:
\begin{verbatim}
Entrada: [1,1,2,3,3,3]
Saída: {1: 2, 2: 1, 3: 3}
\end{verbatim}

\item Escreva uma função que recebe uma lista de listas e retorna uma lista achatada. Qual a complexidade de tempo no pior caso?  Apresente os calculos necessários para chegar a sua resposta.

Exemplo:
\begin{verbatim}
Entrada: [[1, 2], [3, 4], [5]]
Saída: [1, 2, 3, 4, 5]
\end{verbatim}

\item Implemente uma função que verifique se dois conjuntos são disjuntos.

Exemplo:
\begin{verbatim}
Entrada: {1,2,3}, {4,5} => Saída: True
Entrada: {1,2,3}, {3,4} => Saída: False
\end{verbatim}

\item Escreva uma função que rotaciona uma lista para a direita por um número de posições dado. Qual a complexidade de tempo no pior caso?  Apresente os calculos necessários para chegar a sua resposta.

Exemplo:
\begin{verbatim}
Entrada: [1,2,3,4,5], Rotacionar por 2
Saída: [4,5,1,2,3]
\end{verbatim}

\end{enumerate}

\end{document}
