\documentclass[12pt]{article}
\usepackage[utf8]{inputenc}
\usepackage[brazil]{babel}
\usepackage{amsmath, amssymb}
\usepackage{geometry}
\geometry{a4paper, margin=2.5cm}
\title{PCC104 – Prova 2}
\author{Universidade Federal de Ouro Preto}
\date{}

\begin{document}

\maketitle


\begin{enumerate}

\item Resolva a seguinte recorrência:

$T(n) = 2T(n-1) + n$, para $n > 0$, com $T(0) = 1$.

\item Apresente um algoritmo para o qual a expressão da Questão 1 represente o número de operações.

\item Considere a sua implementação do \textit{Recursive Maximum Finder}, \texttt{find\_max(L)} (Lista 4, Recursion, Ex 1). Mostre a árvore de chamadas da sua implementação para \texttt{find\_max([1,5,3,9,2])}

\item Apresente a análise de complexidade completa para a sua implementação do método \texttt{reverse\_string(S)} (Lista 3, Stacks and Queues, Ex. 2)

\item Considere uma matriz \texttt{M}, de dimensão $n \times m$, ordenada. Apresente um algoritmo baseado em busca binária, \texttt{binary\_search(M, x)}, que retorna \texttt{True} se \texttt{x} estiver em \texttt{M} e \texttt{False}, caso contrário. (Sugestão: Faça uma busca binária para encontrar a linha em que o número provavelmente está e, em seguida, execute a busca binária comum nessa linha.)

Exemplos: 
\begin{enumerate}
\item 
  \[ M = \begin{bmatrix}
 1&  2&  5  \\
 6&  7&  8  \\
 11& 15 & 16  \\
 20& 22 & 26  \\
 28& 30&  31  \\
\end{bmatrix} x = 16
\]
$>>$ True

\item
\[ M = \begin{bmatrix}
 11& 15 & 16  \\
 20& 22 & 26  \\
\end{bmatrix} x = 3
\]
$>>$ False

\end{enumerate}


\end{enumerate}

\end{document}
