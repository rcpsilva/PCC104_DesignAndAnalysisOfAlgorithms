\documentclass{article}
\usepackage[utf8]{inputenc}
\usepackage[margin=1.2in]{geometry}
\usepackage{hyperref}

\usepackage{listings}
\usepackage{xcolor}

\definecolor{codegreen}{rgb}{0,0.6,0}
\definecolor{codegray}{rgb}{0.5,0.5,0.5}
\definecolor{codepurple}{rgb}{0.58,0,0.82}
\definecolor{backcolour}{rgb}{0.95,0.95,0.92}

\lstdefinestyle{mystyle}{
    backgroundcolor=\color{backcolour},   
    commentstyle=\color{codegreen},
    keywordstyle=\color{magenta},
    numberstyle=\tiny\color{codegray},
    stringstyle=\color{codepurple},
    basicstyle=\ttfamily\footnotesize,
    breakatwhitespace=false,         
    breaklines=true,                 
    captionpos=b,                    
    keepspaces=true,                 
    numbers=left,                    
    numbersep=5pt,                  
    showspaces=false,                
    showstringspaces=false,
    showtabs=false,                  
    tabsize=2
}

\lstset{style=mystyle}


\usepackage{tikz}
\usetikzlibrary{positioning}

\usepackage{natbib}
\usepackage{graphicx}
\usepackage{amsmath}

\title{\vspace{-2 cm}Federal University of Ouro Preto \\ PCC104 - Project and Analysis of Algorithms \\ Brute Force and Exhaustive Search - Part 2}
\author{Prof. Rodrigo Silva}
%\date{}


\begin{document}

\maketitle

%\section*{Instructions}
% \begin{itemize}
%     \item It is highly recommended to implement the practical activities in C++.
%     \item Make maximum use of algorithms and data structures from the STL library. \url{https://www.geeksforgeeks.org/the-c-standard-template-library-stl/}. 
%     \item Avoid the use of pointers as much as possible, but if needed, use smart pointers \url{https://alandefreitas.github.io/moderncpp/basic-syntax/pointers/smart-pointers/}. 
%     \item When you need a linear data structure, always evaluate the use of the \texttt{vector} class first (\url{https://en.cppreference.com/w/cpp/container/vector})
% \end{itemize}

\section{Recommended Reading}

\begin{itemize}
    \item Chapter 3 - \textit{Introduction to the Design and Analysis of Algorithms (3rd Edition)} - Anany Levitin 
    \item Book - \textit{Problem Solving with Algorithms and Data Structures using C++} (available at: \url{https://runestone.academy/runestone/books/published/cppds/index.html#})
    \item Arrays \url{https://www.interviewcake.com/concept/python/array?}
    %\item LinkedLists \url{https://www.interviewcake.com/concept/python/linked-list?}
    \item Stacks \url{https://www.interviewcake.com/concept/python/stack?}
    \item Queues \url{https://www.interviewcake.com/concept/python/queue?}
    \item Graphs \url{https://www.interviewcake.com/concept/python3/graph}
    \item Book - \textit{Introduction to Programming} - Alan de Freitas (available at \url{http://www.decom.ufop.br/alan/bcc702/livrocpp.pdf})
\end{itemize}


\section{Practical Activities}

\begin{enumerate}
    \item Implement the \textit{BubbleSort} algorithm (See section 3.1)
    \item Implement the \textit{Brute Force String Matching} algorithm (See Section 3.2).
    \item Implement a brute force algorithm to solve the Closest-Pair problem (See section 3.3).
    \item Implement a brute force algorithm to solve the problem of finding the convex hull (See section 3.3)
\end{enumerate}

For each implementation, present the complexity analysis for the worst-case and best-case (if applicable) runtime of the algorithm. This analysis should contain:

\begin{itemize}
    \item A mathematical expression that defines the number of operations (recurrence relation for recursive algorithms or summations for iterative ones) 
    \item Final expression of the cost function
    \item Indication of the efficiency class ($O$ or $\Theta$). The indication of the class must be justified. You can prove it by definition, by limit, or use results demonstrated in the first exercise list (related to Chapter 2 of the book).
\end{itemize}



%\footnotetext{Book - \textit{Introduction to the Design and Analysis of Algorithms (3rd Edition)}}

%\bibliographystyle{plain}
%\bibliography{references}
\end{document}
