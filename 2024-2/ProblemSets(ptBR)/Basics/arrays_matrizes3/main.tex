\documentclass{article}
\usepackage[brazil]{babel}
\usepackage[utf8]{inputenc}
\usepackage{amsmath}
\usepackage{geometry}
\geometry{a4paper, margin=1in}

\title{Lista de Exercícios de Programação: Arrays e Matrizes}
\author{}
\date{}

\begin{document}

\maketitle

\begin{enumerate}

\item \textbf{Controle de Temperatura (Array)}

Uma empresa de meteorologia coleta a temperatura diária durante uma semana (7 dias). Crie um programa que:
\begin{itemize}
    \item Receba as temperaturas de cada dia e armazene em um array.
    \item Calcule a temperatura média da semana.
    \item Informe o dia com a maior temperatura.
\end{itemize}

\item \textbf{Estoque de Produtos (Array)}

Uma loja possui 10 produtos diferentes em seu estoque. Cada produto tem uma quantidade disponível. Crie um programa que:
\begin{itemize}
    \item Receba a quantidade de cada produto e armazene em um array.
    \item Pergunte ao usuário qual produto ele deseja comprar e quantas unidades.
    \item Verifique se é possível realizar a venda. Se sim, atualize o estoque.
\end{itemize}

\item \textbf{Agenda de Consultas (Array)}

Um consultório médico tem uma agenda com 5 horários disponíveis para consultas. Crie um programa que:
\begin{itemize}
    \item Armazene em um array os horários já ocupados.
    \item Pergunte ao usuário qual horário ele deseja agendar.
    \item Verifique se o horário está disponível e, se estiver, marque a consulta.
\end{itemize}

\item \textbf{Notas de Alunos (Matriz)}

Uma escola tem 4 alunos, e cada um tem notas em 3 disciplinas. Crie um programa que:
\begin{itemize}
    \item Receba as notas de cada aluno para cada disciplina e armazene em uma matriz.
    \item Calcule a média de cada aluno.
    \item Informe o aluno com a maior média.
\end{itemize}

\item \textbf{Produção Mensal (Matriz)}

Uma fábrica mede a produção semanal de 3 produtos diferentes durante um mês (4 semanas). Crie um programa que:
\begin{itemize}
    \item Armazene em uma matriz a produção de cada produto em cada semana.
    \item Calcule a produção total de cada produto ao final do mês.
    \item Informe qual foi o produto com maior produção.
\end{itemize}

\item \textbf{Inventário de Biblioteca (Matriz - Índices Flexíveis)}

Uma biblioteca tem 5 estantes, e cada uma possui 4 prateleiras. Crie um programa que:
\begin{itemize}
    \item Armazene em uma matriz a quantidade de livros em cada prateleira.
    \item Utilize loops que começam de índices diferentes (ex.: \texttt{for i = 1} até \texttt{5}) para percorrer as prateleiras de cada estante.
    \item Calcule a quantidade total de livros na biblioteca.
\end{itemize}

\item \textbf{Venda de Ingressos (Matriz - Índices Flexíveis)}

Uma sala de cinema possui 6 fileiras e cada fileira tem 10 assentos. Crie um programa que:
\begin{itemize}
    \item Armazene em uma matriz se cada assento está ocupado ou não (0 para livre, 1 para ocupado).
    \item Utilize \texttt{for} aninhado que comece em índices diferentes (ex.: \texttt{for i = 2} até \texttt{6}) para percorrer os assentos.
    \item Pergunte ao usuário qual assento ele deseja reservar e verifique se está disponível.
\end{itemize}

\item \textbf{Alocação de Salas (Matriz)}

Uma universidade tem 3 prédios, cada um com 5 salas. Crie um programa que:
\begin{itemize}
    \item Armazene em uma matriz a capacidade de cada sala.
    \item Utilize loops que começam de índices diferentes (ex.: \texttt{for i = 1} até \texttt{3}) para percorrer os prédios e salas.
    \item Pergunte ao usuário quantos alunos ele quer alocar e encontre uma sala adequada.
\end{itemize}

\end{enumerate}

\end{document}